%\documentclass[12pt]{amsart}
%~ \documentclass[a4paper,10pt]{report}
\documentclass[a4paper,10pt]{article}
%~ \usepackage[french]{babel}
\usepackage[english]{babel}
\usepackage[utf8]{inputenc} %% problemes d'accents
\usepackage[margin=4cm]{geometry}                % See geometry.pdf to learn the layout options. There are lots.
\geometry{a4paper}                   % ... or a4paper or a5paper or ... 
%\geometry{landscape}                % Activate for for rotated page geometry
%\usepackage[parfill]{parskip}    % Activate to begin paragraphs with an empty line rather than an indent
%\usepackage{microtype}
\usepackage{graphicx}
\usepackage{amssymb}
\setlength{\parindent}{4em}
\setlength{\parskip}{1em}
\renewcommand{\baselinestretch}{1.3}
\usepackage{epstopdf}
\usepackage{tkz-base}
\usepackage[section]{placeins}
\usepackage{float}
%~ \usepackage{ccaption}
%\usepackage[nonumberlist][nopostdot][nogroupskip]{glossaries}
\usepackage[nonumberlist]{glossaries}

\usepackage{multicol}
\usepackage{array, multirow}
%~ \usepackage{caption}
\usepackage{subcaption}
\usepackage{tikz}
\usepackage{pgfplots}
\usetikzlibrary{spy,calc}


\usepackage{varioref}
\usepackage{hyperref}
\hypersetup{
    colorlinks=false,
    pdfborder={0 0 0},
}
\usepackage{cleveref}

\usepackage{graphicx}
\usepackage{amssymb}
\usepackage{epstopdf}

\usepackage{lmodern}
\usepackage{xcolor}
\usepackage{graphicx}
\usepackage{wrapfig}
\usepackage{enumerate}

\usepackage{titlesec}
\usepackage{fancyhdr}
\usepackage{fancybox}
\usepackage{geometry}
\geometry{ hmargin=3cm, vmargin=2.5cm }

\usepackage{listings}
\lstdefinestyle{customc}{
  belowcaptionskip=1\baselineskip,
  breaklines=true,
  frame=L,
  xleftmargin=\parindent,
  language=C,
  showstringspaces=false,
  basicstyle=\footnotesize\ttfamily,
  keywordstyle=\bfseries\color{green!40!black},
  commentstyle=\itshape\color{purple!40!black},
  numberstyle=\color{blue},
  numbers=left, 
  stringstyle=\color{orange},
  tabsize=2,
}
\lstset{style=customc}

\titleformat{\chapter}[display]
{\normalfont\Large\filcenter} {} {1pc} {\titlerule[1pt]
 \vspace{1pc}%
 \Huge}[\vspace{1ex}%
\titlerule]

\DeclareGraphicsRule{.tif}{png}{.png}{`convert #1 `dirname #1`/`basename #1 .tif`.png}
\def\rightenv{13.2cm}

\renewcommand{\headrulewidth}{0.4pt}   % trait horizontal en dessous de l'entete
\renewcommand{\footrulewidth}{0.4pt}     % trait horizontal au dessus du pied de page
\renewcommand{\baselinestretch}{1.1} % augmente l'interligne a 1.1

\topmargin=-7mm
\headheight=15pt
\headsep=10mm

%~ \newtheorem{theorem}{Théorème}

\newcommand{\scal}[2]{\left\langle {#1} \middle| {#2} \right\rangle}
\newcommand{\console}[1]{\colorbox{black}{\begin{minipage}[c]{1\linewidth}\textcolor{white}{\texttt{#1}}\end{minipage}}}
 
\newcommand{\annexe}[1]{%
	\newpage
	\refstepcounter{section}%
	\phantomsection%
	\addcontentsline{toc}{section}{\appendixname~{\thesection}: #1}%
	\section*{\appendixname~{\thesection}: #1}
	}

\usepackage{algorithm}
\usepackage{algorithmic}
\newcommand{\switch}{%
  \mathcode`+=\numexpr\mathcode`+ + "1000\relax % turn + into a relation
  \mathcode`*=\numexpr\mathcode`* + "1000\relax
}	

\newtheorem{algoxxx}{\bf Algorithm}
\newcommand{\algorithmautorefname}{Algorithm}
%~ \newtheorem{algoxxx}{\bf Algorithme}
%~ \newcommand{\algorithmautorefname}{Algorithme}
%~ \floatname{algorithm}{Algorithme }
%~ \algsetup{indent=2em}
%~ \renewcommand{\algorithmicrequire}{\textbf{Nécessite :}}
%~ \renewcommand{\algorithmicensure}{\textbf{Verifie :}}
%~ \renewcommand{\algorithmiccomment}[1]{\{#1\}}
%~ \renewcommand{\algorithmicend}{\textbf{fin}}
%~ \renewcommand{\algorithmicif}{\textbf{si}}
%~ \renewcommand{\algorithmicthen}{\textbf{alors}}
%~ \renewcommand{\algorithmicelse}{\textbf{sinon}}
%~ \renewcommand{\algorithmicelsif}{\algorithmicelse\ \algorithmicif}
%~ \renewcommand{\algorithmicendif}{\algorithmicend\ \algorithmicif}
%~ \renewcommand{\algorithmicfor}{\textbf{pour}}
%~ \renewcommand{\algorithmicforall}{\textbf{pour tout}}
%~ \renewcommand{\algorithmicdo}{\textbf{faire}}
%~ \renewcommand{\algorithmicendfor}{\algorithmicend\ \algorithmicfor}
%~ \renewcommand{\algorithmicwhile}{\textbf{tant que}}

\newlength\myindent
\setlength\myindent{2em}
\newcommand\bindent{%
  \begingroup
  \setlength{\itemindent}{\myindent}
  \addtolength{\algorithmicindent}{\myindent}
}
\newcommand\eindent{\endgroup}

\newcommand{\MONTHENG}{%
  \ifcase\month% 0
    \or January% 1
    \or February% 2
    \or March% 3
    \or April% 4
    \or May% 5
    \or June% 6
    \or July% 7
    \or August% 8
    \or September% 9
    \or October% 10
    \or November% 11
    \or December% 12
  \fi}
\newcommand{\MONTHFR}{%
  \ifcase\month% 0
    \or Janvier% 1
    \or Février% 2
    \or Mars% 3
    \or Avril% 4
    \or Mai% 5
    \or Juin% 6
    \or Juillet% 7
    \or Aout% 8
    \or Septembre% 9
    \or Octobre% 10
    \or Novembre% 11
    \or Décembre% 12
  \fi}
\newcommand{\YEAR}{\number\year}


\pgfplotsset{compat=1.15}
