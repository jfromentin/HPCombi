\appendix

\annexe{Basics of \glstext{gpu} architecture}
\label{principegpu}
%\subsection{Differences between \glstext{gpu} and \glstext{cpu}}

\autoref{archi} and \autoref{comp-cpu-gpu} show the main differences between \glspl{cpu} and \glspl{gpu}. 
A \gls{gpu} can contain up to several thousand \glspl{alu}, whereas \gls{cpu} contain a few dozens.
The \gls{gpu} \gls{gm} (comparable to \gls{ram} memory on \glspl{cpu}) is on the same chip as the \glspl{alu}, thus allowing high bandwidth.
\glspl{cpu} often operate at faster clock speeds than \glspl{gpu}.

\begin{table}[H]
\begin{multicols}{2}
\begin{itemize}
\item [\gls{cpu} :]
\item \gls{cpu} cores : 2 to 72,
\item \gls{ram} bandwidth : 8 to 540 Go/s,
\item Clock frequency : 1~200 to 3~500 MHz.
\end{itemize}
\columnbreak
\begin{itemize}
\item [\gls{gpu} :]
\item \glstext{cuda} cores : 250 to 5~120,
\item \Gls{gm} bandwidth : 160 to 900 Go/s,
\item Clock frequency : 745 to 1~480 MHz.
\end{itemize}
\end{multicols}
\caption{Comparison of typical specifications of \glspl{cpu} and \glspl{gpu}}
\label{comp-cpu-gpu}
\end{table}


\oneimage{archi}{Schematic of \gls{cpu} and \gls{gpu} architectures . \footnotesize{(source : \cite{doccuda})}}{archi}{0.7}

%\subsection{\glstext{gpu}}
\label{partgpu}

\gls{nvidia} \gls{gpu} are organized as follows \cite{doccuda} :
\begin{enumerate}

\item A \gls{gpu} is composed of several \gls{smx} (15 on \gls{k40}).

\item A \gls{smx} executes several \glspl{th} \glspl{bloc}.
The number of \glspl{th} \glspl{block} executed on a \gls{smx} is not directly available to the programmer.
\Gls{shared} and register usage by the kernel are the two factor limiting the number of \glspl{block} executed on a \gls{smx}.
On the Kepler architecture 65~536 registers and 48 Kbytes of \gls{shared} are available on a \gls{smx}.

\item The L2 cache is common to all \gls{smx} and the L1 cache is common to all \glspl{th} within a \gls{smx}.

\item A \glstext{cbloc} contains several \glspl{th} (up to 1~024).
The \gls{block} size is set by the programmer in 3 dimensions. It is a parameter to optimize.
\Gls{shared} is a memory bank shared over all threads from a \gls{block}.

\item \Glspl{block} are organized in a 3 dimensions grid also set by the programmer. 
\autoref{grid-block} illustrates the \glspl{th} organization in a grid of \glspl{block}.

\item A \gls{warp} is a group of 32 \glspl{th} of a block.
All \gls{th} in a \gls{warp} execute the same instructions according to the \gls{simt} protocol.

\end{enumerate}
\bigskip

%Les \gls{gpu} sont utilisés en tant que \glspl{coproc}, c'est-à-dire que le code est exécuté par le \gls{cpu}, 
%appelé \gls{host}, et ce dernier sous-traite les \glspl{kernel} aux \gls{gpu}. 
%Le programmeur doit gérer les transferts de données du \gls{cpu} vers les \gls{gpu} et des \gls{gpu} vers le \gls{cpu}.

\oneimage{gpu-grid-block}{Schematic of a grid of \glspl{bloc} \footnotesize{(source : \cite{doccuda})}}{grid-block}{1}

%La gestion des accès aux différentes mémoires est cruciale : les mémoires distantes (\gls{ram} \gls{cpu} et \gls{mg} \gls{gpu}) sont les plus lentes. 
%Le \autoref{tabdebitmemoire} donne les ordres de grandeur des débits des différentes mémoires.
%Il faut donc utiliser le plus possible les mémoires locales (\glspl{reg} et \gls{shared}), en particulier en limitant les transferts \gls{cpu}-\gls{gpu}. 
%Les accès à la \gls{mg} doivent se faire idéalement de manière contiguë, 
%car les lectures cohérentes sont près de 20 fois plus rapides que les lectures aléatoires \cite{coursGPUonera}. 
%Les accès à la \gls{mg} doivent également être recouverts par des calculs pour masquer les latences d'accès.

%~ \begin{table}
%~ \centering
%~ \begin{tabular}{p{3cm}|>{\centering\arraybackslash}p{3cm}>{\centering\arraybackslash}p{3cm}>{\centering\arraybackslash}p{3cm}}
%~ \hline
%~ \gls{cpu}-\gls{gpu} & \Gls{mg} & \Gls{shared} & \Glspl{reg} \\
%~ \hline
%~ 10-150 Go/s & 160-900 Go/s & 1 To/s & 10 To/s \\
%~ \hline
%~ \end{tabular}
%~ \caption{Ordres de grandeur des débits mémoires. }
%~ \label{tabdebitmemoire}
%~ \end{table}

%%%%%%%%%%%%%%%%%%%%%%%%%%%%%%%%%%%%%%%%%%%%%%%%%%%%%%%%%%%%%%%%%%%%%%%%%%%%%%%%%%%%%%%%%%%%%%%%%%%%%%%%%%%%%%%%%%%%%%%%%%%%%%%%%%%%%%%%%%%%%%%%%%%%%%%%%%%%%%%%%%%%%%%%%%%%%%

\annexe{Specifications of the \glstext{gpu}s and \glstext{cpu}s used in this study}
\label{spec}
\centering
\begin{tabular}{p{7cm}|>{\centering\arraybackslash}p{3.2cm}>{\centering\arraybackslash}p{3.2cm}}

& \textbf{\gls{gpu} \glstext{k40}} \cite{dock40} & \textbf{2$\times$\gls{cpu} \glstext{e52620v2}} \cite{doce5v2}\\

\hline
Launch year & 2013 & 2012 \\

Architecture & Kepler & Ivy Bridge \\
\hline
%~ \gls{cuda} Cores/\gls{cpu} Cores & 2880 & 2$\times$6 \\
\gls{smx}/\gls{cpu} Cores & 15 & 2$\times$6 \\

\gls{cpu} threads & X & 2$\times$12 \\

\glstext{cuda} Cores & 2880 & X \\

Frequency (GHz) & 0.74 & 2.1 \\
\hline
\Gls{gm}/\gls{ram} size (Go) & 12 & 64 \\

\Gls{gm}/\gls{ram} max bandwidth (Go/s) & 288 & 51 \\
\hline
TDP (W) & 235 & 80 \\

\end{tabular}

